% Tento soubor nahraďte vlastním souborem s obsahem práce.
%=========================================================================
% Autoři: Michal Bidlo, Bohuslav Křena, Jaroslav Dytrych, Petr Veigend a Adam Herout 2019
\chapter{Úvod}
Tento dokument popisuje riešenie automatického získavania a spracovania noviniek zo vzdialeného serveru s podporou pre zabezpečenú komunikáciu. 

Táto kapitola približuje základné a zjednodušené fungovanie jednotlivých častí implementácie.

\chapter{Základné informácie}
\label{basic_info}
Naimplementovaná aplikácia sa pripája k požadovanému zdroju, stiahne ATOM alebo RSS 2.0 dáta, spracuje ich a vypíše užívateľovi na štandarný výstup. Pri HTTPS komunikácii, aplikácia podporuje SSL/TLS pre overenie certifikátu zdroju.
Požadovaný zdroj môže byť vložený pomocou jednej URL alebo pomocou súboru s viacerými URL pri spustení aplikácie.
Vždy sa vypíše názov zdroju a názvy jednotlivých príspevkov.
Užívateľ si môže nechať vypísať dátum publikovania, URL na konkrétny príspevok a autora príspevku.

\subsection*{Rozšírenie}
\begin{itemize}
  \item{Rôzne návratové hodnoty pri chybách, definovaných v {\tt../src/error.h}.}
  \item{Lokalizácia správ aplikácie do čestiny a angličtiny. Prepínač je v {\tt../src/error.h}.}
\end{itemize}

\subsection*{Obmedzenia}
\begin{itemize}
  \item{Kompilácia zdrojového kódu sa musí spraviť pomocou GNU Make.}
  \item{Aplikácia je vyvinutá pre UNIX OS.}
\end{itemize}

\chapter{Návrh aplikácie}
\subsection*{Stromová štruktúra}
\dirtree{%
.1 bin.
.1 src.
.2 atom.c.
.2 atom.h.
.2 \dots.
.1 tests.
.2 examples.
.3 FIT-cert.
.3 \dots.
.2 \dots.
.1 Makefile.
.1 manual.pdf.
.1 README.md.
}
\hfill \break
\noindent Význam priečinkov aplikácie:
\begin{itemize}
  \item{priečinok {\tt bin} bude po kompilácii aplikácie obsahovať objektové súbory zdrojových súborov}
  \item{priečinok {\tt src} obsahuje zdrojové súbory implementácie aplikácie}
  \item{priečinok {\tt tests} obsahuje zdrojové súbory testov aplikácie}
\end{itemize}

\subsection*{Lokalizácia}
Nad rámec zadania bola naimplementovaná dvojjazyčná mutácia aplikácie. Lokalizácia poskytuje českú a anglickú jazykovú mutáciu.

Predvolený jazykom je čestina a zmena je možná prenastavením hodnoty \bold{LANG} v {\tt ../src/error.h} na 0 (nula). Toto nastavenie ovplyvňuje preklad časti výstupu pre užívateľa, ako napr. {\it Autor:} na {\it Author:} pri výpise autora článku. Ako ďalší vplyv tohto nastavenia je preklad všetkých chybových hláškok, viac \ref{err_label}.

Chybové hlášky sú definové v {\tt ../src/error.h}, keďže veľmi blízko súvisia s oznamovaním chýb.

\chapter{Implementácia}
Naimplementovaná čítačka noviniek je napísaná v programovacom jazyku C. Pri vývoji aplikácie boli použité štandardné knižnice jazyka C ({\it stdio, stdlib, string, getopt, time\dots}), sieťové knižnice({\tt sys/*, netdb}), knižnicu openssl pre TLS/SSL a knižnicu libxml pre spracovanie XML odpovede.
\subsection*{Vstupné parametre}
Spracovanie uživatelského vstupu má na starosti {\tt ../src/parameters.c}. Kontroluje správnosť kombinácii vstupných parametrov, ich požadovaného správania podľa zadania projektu.
\subsubsection{Možnosti}
\begin{itemize}
  \item{URL so schématom}
  \item{-f {\it cesta-k-suboru-s-urls}}
  \item{-c {\it cesta-k-suboru-pertifikatu}}
  \item{-C {\it cesta-k-priečinku-pertifikatov}}
  \item{prepínače zobrazenia doplňujúcich dát článku: -a, -u, -T}
\end{itemize}
\subsubsection{Kombinácie}
Aplikácia nepodporuje kombinácie:
\begin{itemize}
  \item{URL a -f {\it cesta-k-suboru-s-urls}}
  \item{-c {\it cesta-k-suboru-pertifikatu} a -C {\it cesta-k-priečinku-pertifikatov}}
  \item{mnohonásobné použitie prepínačov alebo URL}
\end{itemize}

Validné príklady použitia sú k nahliadnutiu v kapitole \ref{examples}.

\subsection*{Spracovanie URL}
URL môže byť užívateľom vložená do aplikácie dvomi spôsobmi:
\begin{itemize}
  \item{pri spustení cez terminál}
  \item{vložený do súboru, ktorý je predaný aplikácii s parametrom {\tt -f}}
  \item{mnohonásobné použitie prepínačov alebo URL}
\end{itemize}

O spracovanie URL z príkazového riadku a súboru sa starajú {\tt ../src/host.c}, konkrétne funkcie {\it parse\_url()} a {\it read\_urls()}. 

Spracovanie URL prebieha pomocou regulárneho výrazu\cite{regexMan}, ktorý bol navrhnutý tak, aby pri použití na URL vyhľadal a jednotlivé časti (skupiny) URL. A to:
\begin{itemize}
  \item{http:// alebo https://}
  \item{www.}
  \item{{\it domenovy-nazov-serveru.domenu-najvyssej-urovne}}
  \item{{\it :port}}
  \item{{\it cesta-k-suboru}}
\end{itemize}

Súbor pre parameter {\tt -f} môže obsahovať jednu URL na každom riadku, ďalej ak má na začiatku znak '\#', tento riadok je braný ako komentár. Každý nekomentový riadok je spracovaný funkciou {\it parse\_url()}. Pri chybnom formáte URL, je na štandardný chybový výstup vypísaná chyba a aplikácia pokračuje ďalej vo vykonávaní implementácie.

\subsection*{TCP komunikácia}
Implementácia TCP časti sa nachádza v {\tt ../src/tcp\_communication.c}, ktorého podstata vychádza z prednášky a jej príkladov predmetu ISA, konkrétne prednášku, docenta Matoušku, Pokrocilé programování sítí TCP/IP\cite{Matousek}. 

\subsubsection{HTTP a HTTPS}
HTTP a HTTPS komunikácia sú implementované v zdrojový súboroch 
\\{\tt ../src/http\_communication.c} a {\tt ../src/https\_communication.c}.Funkcie oboch týchto komunikácií sú volané, po prevedí predošlých úkonov, napr. naviazenie TLS spojenia pri HTTPS, zasielajú HTTP 1.0 dotaz na cieľovú destináciu. Inšpiráciou pre HTTP dotaz, po menšej úprave, bola časť kódu z referečnej knihy predmetu ISA\cite{Winkle}.
\subsubsection{Prijímanie a uloženie dát}
Prijímanie a uloženie je taktiež naimplementované v súboroch
\\{\tt ../src/http\_communication.c} a {\tt ../src/https\_communication.c}. Výsledným produktom, pri úspešnej odpovedi cieľového serveru a prijatí všetkých dát, je dynamicky alokovaný reťazec, ktorý sa ďalej spracováva.

Ako úspešnú odpoveď, implementácia považuje odpoveď so statusom {\tt 200}. Všetky ostatné\cite{ibmHttp} sú brané ako neúspešná odpoveď a spracovanie dát sa ukončuje. Užívateľ si môže v chybovej hláške nájsť.

Inšpiráciou pre prijímanie odpovedí z sielového serveru boli už spomínaná prednáška docenta Matoušku\cite{Matousek} a refrečná kniha predmetu ISA\cite{Winkle}.

Ukážky kódu z refrečnej knihy predmetu ISA\cite{Winkle}, po úpravách a rozšíreniach, boli taktiež použité, ako inšpirácia pre pre TLS komunikáciu a prácu s certifikátmi vo funkciách súboru {\tt ../src/https\_communication.c}

\subsection*{Spracovanie XML}
Hlavné spracovanie XML odpovede sa deje v {\tt ../src/feed.c}, kde sa reťazec konvertuje do XML objektu pomocou funkcie knižnice {\it libxml}. Po úspešnom prekonvertovaní a nájdení koreňa XML, nasleduje spracovanie podľa typu štruktúry XML. Pri spracovaní sa vypíše názov zdroja a jednotlivé názvy článkov. 

Ak užívateľ zadal príslušné paramatri a odpoveď obsahuje také údaje, aplikácia vypíše informácie o autorovi, dátume vydania a URI článku. 

V prípade, že boli použité dodatočné parametre, články sú oddelené jedným prázdnym riadkom. Informácie o článku sú v poradí:
\begin{enumerate}
  \item{názov článku}
  \item{meno autora článku}
  \item{URI článku}
  \item{dátum publikovania článku}
\end{enumerate}

Ak nejaká dodatočná informácia chýba alebo nebola vybraná, preskočí sa a vypíše sa ďalšia v poradí.

\subsubsection{Atom a RSS}
Spracovanie je implementované v {\tt ../src/atom.c} a v {\tt ../src/rss.c}

\subsection*{Oštrenie chýb}
\label{err_label}
Úspešný chod aplikácie vráti hodnotu 0. Pri chybách, aplikácia vráti príslušnú chybovú hodnotu, definovanú v {\tt ../src/error.c}

Výpis návratových hodnôt:
\begin{itemize}
  \item{0 - úspech}
  \item{1 - všeobecná chyba}
  \item{2 - chyba pri kompilácii regulárného výrazu}
  \item{10 - žiadne vstupné parametre}
  \item{11 - neznámy parameter}
  \item{12 - niekoľkonásobné použitie rovnakého parametru}
  \item{13 - nepovolená kombinácia parametrov}
  \item{14 - viaceré zdroje cieľových destinácii}
  \item{15 - zlý formát URL}
  \item{16 - chýbajúca cesta k súboru na cieľovej destinácii}
  \item{17 - nezadaný zdroj cieľovej destinácie}
  \item{18 - chýbajúci/neznámy súbor s certifikátom}
  \item{19 - chýbajúci/neznámy priečinok s certifikátmi}
  \item{20 - nepodarilo sa preložiť URL na IP adresu}
  \item{21 - neplatná cieľová destinácia}
  \item{22 - vypršal čas na odpoveď}
  \item{23 - vytváranie schránky zlyhalo}
  \item{24 - nepodarilo sa naviazať spojenie s cieľovou destináciou}
  \item{30 - chyba pri čítaní súboru}
  \item{31 - súbor je prázdny alebo neobsahuje valídne URL}
  \item{40 - neúspešná odpoveď od cieľového serveru, http status > 200}
  \item{50 - chyba pri vytváraní CTX kontextu}
  \item{51 - chyba pri vytváraní SSL objketu}
  \item{52 - chyba pri naväzovaní SSL komunikácie}
  \item{53 - chyba pri identifikovaní serveru}
  \item{54 - nepodarilo sa získať certifikát cieľovej destinácie}
  \item{60 - XML odpoveď je prázdna}
  \item{61 - spracovanie XML stromu zlyhalo}
  \item{62 - chyba pri konvertovaní XML na XML objekt}
  \item{70 - chyba pri načítaní súboru certifikátu}
  \item{71 - chyba pri načítaní priečinku certifikátov}
  \item{72 - chyba pri nastavovaní predvoleného priečinku certifikátov}
  \item{73 - overenie certifikátu zlyhalo}
\end{itemize}

\chapter{Príklady použitia}
\label{examples}

%===============================================================================
