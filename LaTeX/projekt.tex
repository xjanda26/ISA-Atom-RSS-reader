%==============================================================================
% tento soubor pouzijte jako zaklad
% this file should be used as a base for the thesis
% Autoři / Authors: 2008 Michal Bidlo, 2019 Jaroslav Dytrych
% Kontakt pro dotazy a připomínky: sablona@fit.vutbr.cz
% Contact for questions and comments: sablona@fit.vutbr.cz
%==============================================================================
% kodovani: UTF-8 (zmena prikazem iconv, recode nebo cstocs)
% encoding: UTF-8 (you can change it by command iconv, recode or cstocs)
%------------------------------------------------------------------------------
% zpracování / processing: make, make pdf, make clean
%==============================================================================
% Soubory, které je nutné upravit nebo smazat: / Files which have to be edited or deleted:
%   projekt-20-literatura-bibliography.bib - literatura / bibliography
%   projekt-01-kapitoly-chapters.tex - obsah práce / the thesis content
%   projekt-01-kapitoly-chapters-en.tex - obsah práce v angličtině / the thesis content in English
%   projekt-30-prilohy-appendices.tex - přílohy / appendices
%   projekt-30-prilohy-appendices-en.tex - přílohy v angličtině / appendices in English
%==============================================================================
\documentclass[slovak]{fitthesis} % bez zadání - pro začátek práce, aby nebyl problém s překladem
%\documentclass[english]{fitthesis} % without assignment - for the work start to avoid compilation problem
%\documentclass[zadani]{fitthesis} % odevzdani do wisu a/nebo tisk s barevnými odkazy - odkazy jsou barevné
%\documentclass[english,zadani]{fitthesis} % for submission to the IS FIT and/or print with color links - links are color
%\documentclass[zadani,print]{fitthesis} % pro černobílý tisk - odkazy jsou černé
%\documentclass[english,zadani,print]{fitthesis} % for the black and white print - links are black
%\documentclass[zadani,cprint]{fitthesis} % pro barevný tisk - odkazy jsou černé, znak VUT barevný
%\documentclass[english,zadani,cprint]{fitthesis} % for the print - links are black, logo is color
% * Je-li práce psaná v anglickém jazyce, je zapotřebí u třídy použít 
%   parametr english následovně:
%   If thesis is written in English, it is necessary to use 
%   parameter english as follows:
%      \documentclass[english]{fitthesis}
% * Je-li práce psaná ve slovenském jazyce, je zapotřebí u třídy použít 
%   parametr slovak následovně:
%   If the work is written in the Slovak language, it is necessary 
%   to use parameter slovak as follows:
%      \documentclass[slovak]{fitthesis}
% * Je-li práce psaná v anglickém jazyce se slovenským abstraktem apod., 
%   je zapotřebí u třídy použít parametry english a enslovak následovně:
%   If the work is written in English with the Slovak abstract, etc., 
%   it is necessary to use parameters english and enslovak as follows:
%      \documentclass[english,enslovak]{fitthesis}

% Základní balíčky jsou dole v souboru šablony fitthesis.cls
% Basic packages are at the bottom of template file fitthesis.cls
% zde můžeme vložit vlastní balíčky / you can place own packages here
\usepackage{dirtree}

% Kompilace po částech (rychlejší, ale v náhledu nemusí být vše aktuální)
% Compilation piecewise (faster, but not all parts in preview will be up-to-date)
% \usepackage{subfiles}

% Nastavení cesty k obrázkům
% Setting of a path to the pictures
%\graphicspath{{obrazky-figures/}{./obrazky-figures/}}
%\graphicspath{{obrazky-figures/}{../obrazky-figures/}}

%---rm---------------
\renewcommand{\rmdefault}{lmr}%zavede Latin Modern Roman jako rm / set Latin Modern Roman as rm
%---sf---------------
\renewcommand{\sfdefault}{qhv}%zavede TeX Gyre Heros jako sf
%---tt------------
\renewcommand{\ttdefault}{lmtt}% zavede Latin Modern tt jako tt

% vypne funkci šablony, která automaticky nahrazuje uvozovky,
% aby nebyly prováděny nevhodné náhrady v popisech API apod.
% disables function of the template which replaces quotation marks
% to avoid unnecessary replacements in the API descriptions etc.
\csdoublequotesoff



\usepackage{url}


% =======================================================================
% balíček "hyperref" vytváří klikací odkazy v pdf, pokud tedy použijeme pdflatex
% problém je, že balíček hyperref musí být uveden jako poslední, takže nemůže
% být v šabloně
% "hyperref" package create clickable links in pdf if you are using pdflatex.
% Problem is that this package have to be introduced as the last one so it 
% can not be placed in the template file.
\ifWis
\ifx\pdfoutput\undefined % nejedeme pod pdflatexem / we are not using pdflatex
\else
  \usepackage{color}
  \usepackage[unicode,colorlinks,hyperindex,plainpages=false,pdftex]{hyperref}
  \definecolor{hrcolor-ref}{RGB}{223,52,30}
  \definecolor{hrcolor-cite}{HTML}{2F8F00}
  \definecolor{hrcolor-urls}{HTML}{092EAB}
  \hypersetup{
	linkcolor=hrcolor-ref,
	citecolor=hrcolor-cite,
	filecolor=magenta,
	urlcolor=hrcolor-urls
  }
  \def\pdfBorderAttrs{/Border [0 0 0] }  % bez okrajů kolem odkazů / without margins around links
  \pdfcompresslevel=9
\fi
\else % pro tisk budou odkazy, na které se dá klikat, černé / for the print clickable links will be black
\ifx\pdfoutput\undefined % nejedeme pod pdflatexem / we are not using pdflatex
\else
  \usepackage{color}
  \usepackage[unicode,colorlinks,hyperindex,plainpages=false,pdftex,urlcolor=black,linkcolor=black,citecolor=black]{hyperref}
  \definecolor{links}{rgb}{0,0,0}
  \definecolor{anchors}{rgb}{0,0,0}
  \def\AnchorColor{anchors}
  \def\LinkColor{links}
  \def\pdfBorderAttrs{/Border [0 0 0] } % bez okrajů kolem odkazů / without margins around links
  \pdfcompresslevel=9
\fi
\fi
% Řešení problému, kdy klikací odkazy na obrázky vedou za obrázek
% This solves the problems with links which leads after the picture
\usepackage[all]{hypcap}

% Informace o práci/projektu / Information about the thesis
%---------------------------------------------------------------------------
\projectinfo{
  %Prace / Thesis
  project={TD},            %typ práce BP/SP/DP/DR  / thesis type (SP = term project)
  year={2022},             % rok odevzdání / year of submission
  date=\today,             % datum odevzdání / submission date
  %Nazev prace / thesis title
  title.cs={Čtečka novinek ve formátu Atom a RSS s podporou TLS},  % název práce v češtině či slovenštině (dle zadání) / thesis title in czech language (according to assignment)
  title.en={News reader in Atom and RSS format with TLS support}, % název práce v angličtině / thesis title in english
  %title.length={14.5cm}, % nastavení délky bloku s titulkem pro úpravu zalomení řádku (lze definovat zde nebo níže) / setting the length of a block with a thesis title for adjusting a line break (can be defined here or below)
  %sectitle.length={14.5cm}, % nastavení délky bloku s druhým titulkem pro úpravu zalomení řádku (lze definovat zde nebo níže) / setting the length of a block with a second thesis title for adjusting a line break (can be defined here or below)
  %dectitle.length={14.5cm}, % nastavení délky bloku s titulkem nad prohlášením pro úpravu zalomení řádku (lze definovat zde nebo níže) / setting the length of a block with a thesis title above declaration for adjusting a line break (can be defined here or below)
  %Autor / Author
  author.name={Adam},   % jméno autora / author name
  author.surname={Janda},   % příjmení autora / author surname 
  %author.title.p={Bc.}, % titul před jménem (nepovinné) / title before the name (optional)
  %author.title.a={Ph.D.}, % titul za jménem (nepovinné) / title after the name (optional)
  %Ustav / Department
  department={UIFS}, % doplňte příslušnou zkratku dle ústavu na zadání: UPSY/UIFS/UITS/UPGM / fill in appropriate abbreviation of the department according to assignment: UPSY/UIFS/UITS/UPGM
  % Školitel / supervisor
  supervisor.name={Libor},   % jméno školitele / supervisor name 
  supervisor.surname={Polčák},   % příjmení školitele / supervisor surname
  supervisor.title.p={Ing.},   %titul před jménem (nepovinné) / title before the name (optional)
  supervisor.title.a={Ph.D.},    %titul za jménem (nepovinné) / title after the name (optional)
  % Klíčová slova / keywords
  keywords.cs={Sem budou zapsána jednotlivá klíčová slova v českém (slovenském) jazyce, oddělená čárkami.}, % klíčová slova v českém či slovenském jazyce / keywords in czech or slovak language
  keywords.en={Sem budou zapsána jednotlivá klíčová slova v anglickém jazyce, oddělená čárkami.}, % klíčová slova v anglickém jazyce / keywords in english
  %keywords.en={Here, individual keywords separated by commas will be written in English.},
  % Abstrakt / Abstract
  abstract.cs={Do tohoto odstavce bude zapsán výtah (abstrakt) práce v českém (slovenském) jazyce.}, % abstrakt v českém či slovenském jazyce / abstract in czech or slovak language
  abstract.en={Do tohoto odstavce bude zapsán výtah (abstrakt) práce v anglickém jazyce.}, % abstrakt v anglickém jazyce / abstract in english
  %abstract.en={An abstract of the work in English will be written in this paragraph.},
  % Prohlášení (u anglicky psané práce anglicky, u slovensky psané práce slovensky) / Declaration (for thesis in english should be in english)
  declaration={Prohlašuji, že jsem tuto bakalářskou práci vypracoval samostatně pod vedením pana X...
Další informace mi poskytli...
Uvedl jsem všechny literární prameny, publikace a další zdroje, ze kterých jsem čerpal.},
  %declaration={I hereby declare that this Bachelor's thesis was prepared as an original work by the author under the supervision of Mr. X
% The supplementary information was provided by Mr. Y
% I have listed all the literary sources, publications and other sources, which were used during the preparation of this thesis.},
  % Poděkování (nepovinné, nejlépe v jazyce práce) / Acknowledgement (optional, ideally in the language of the thesis)
  acknowledgment={V této sekci je možno uvést poděkování vedoucímu práce a těm, kteří poskytli odbornou pomoc
(externí zadavatel, konzultant apod.).},
  %acknowledgment={Here it is possible to express thanks to the supervisor and to the people which provided professional help
%(external submitter, consultant, etc.).},
  % Rozšířený abstrakt (cca 3 normostrany) - lze definovat zde nebo níže / Extended abstract (approximately 3 standard pages) - can be defined here or below
  %extendedabstract={Do tohoto odstavce bude zapsán rozšířený výtah (abstrakt) práce v českém (slovenském) jazyce.},
  %extabstract.odd={true}, % Začít rozšířený abstrakt na liché stránce? / Should extended abstract start on the odd page?
  %faculty={FIT}, % FIT/FEKT/FSI/FA/FCH/FP/FAST/FAVU/USI/DEF
  faculty.cs={Fakulta informačních technologií}, % Fakulta v češtině - pro využití této položky výše zvolte fakultu DEF / Faculty in Czech - for use of this entry select DEF above
  faculty.en={Faculty of Information Technology}, % Fakulta v angličtině - pro využití této položky výše zvolte fakultu DEF / Faculty in English - for use of this entry select DEF above
  department.cs={Ústav matematiky}, % Ústav v češtině - pro využití této položky výše zvolte ústav DEF nebo jej zakomentujte / Department in Czech - for use of this entry select DEF above or comment it out
  department.en={Institute of Mathematics} % Ústav v angličtině - pro využití této položky výše zvolte ústav DEF nebo jej zakomentujte / Department in English - for use of this entry select DEF above or comment it out
}

% Rozšířený abstrakt (cca 3 normostrany) - lze definovat zde nebo výše / Extended abstract (approximately 3 standard pages) - can be defined here or above
%\extendedabstract{Do tohoto odstavce bude zapsán výtah (abstrakt) práce v českém (slovenském) jazyce.}
% Začít rozšířený abstrakt na liché stránce? / Should extended abstract start on the odd page?
%\extabstractodd{true}

% nastavení délky bloku s titulkem pro úpravu zalomení řádku - lze definovat zde nebo výše / setting the length of a block with a thesis title for adjusting a line break - can be defined here or above
\titlelength{12.5cm}
% nastavení délky bloku s druhým titulkem pro úpravu zalomení řádku - lze definovat zde nebo výše / setting the length of a block with a second thesis title for adjusting a line break - can be defined here or above
%\sectitlelength{14.5cm}
% nastavení délky bloku s titulkem nad prohlášením pro úpravu zalomení řádku - lze definovat zde nebo výše / setting the length of a block with a thesis title above declaration for adjusting a line break - can be defined here or above
%\dectitlelength{14.5cm}

% řeší první/poslední řádek odstavce na předchozí/následující stránce
% solves first/last row of the paragraph on the previous/next page
\clubpenalty=10000
\widowpenalty=10000

% checklist
\newlist{checklist}{itemize}{1}
\setlist[checklist]{label=$\square$}

% Nechcete-li, aby se u oboustranného tisku roztahovaly mezery pro zaplnění stránky, odkomentujte následující řádek / If you do not want enlarged spacing for filling of the pages in case of duplex printing, uncomment the following line
% \raggedbottom

\begin{document}
  % Vysazeni titulnich stran / Typesetting of the title pages
  % ----------------------------------------------
  \maketitle
  % Obsah
  % ----------------------------------------------
  \setlength{\parskip}{0pt}

  {\hypersetup{hidelinks}\tableofcontents}
  
  % Seznam obrazku a tabulek (pokud prace obsahuje velke mnozstvi obrazku, tak se to hodi)
  % List of figures and list of tables (if the thesis contains a lot of pictures, it is good)
  \ifczech
    \renewcommand\listfigurename{Seznam obrázků}
  \fi
  \ifslovak
    \renewcommand\listfigurename{Zoznam obrázkov}
  \fi
  % {\hypersetup{hidelinks}\listoffigures}
  
  \ifczech
    \renewcommand\listtablename{Seznam tabulek}
  \fi
  \ifslovak
    \renewcommand\listtablename{Zoznam tabuliek}
  \fi
  % {\hypersetup{hidelinks}\listoftables}

  \ifODSAZ
    \setlength{\parskip}{0.5\bigskipamount}
  \else
    \setlength{\parskip}{0pt}
  \fi

  % vynechani stranky v oboustrannem rezimu
  % Skip the page in the two-sided mode
  \iftwoside
    \cleardoublepage
  \fi

  % Text prace / Thesis text
  % ----------------------------------------------

  % Tento soubor nahraďte vlastním souborem s obsahem práce.
%=========================================================================
% Autor: Adam Janda, xjanda26
\chapter{Úvod}
Tento dokument popisuje riešenie automatického získavania a spracovania noviniek zo vzdialeného serveru s podporou pre zabezpečenú komunikáciu. 

Nasledujúcich kapitolách sa priblížia základné a zjednodušené fungovanie jednotlivých častí implementácie. Najprv sa uzrejmia základné informácie, odlišnosti od zadania, obmedzenia, spôsob kompilácie celého projektu/aplikácie a testy aplikácie. Potom sa spomenie stronová štruktúra súborov a priečinkov, a lokalizácia. Nakoniec sa v stručnosti vysvetlí implementácia, chybové ošetrenia a príklady použitia.

\chapter{Základné informácie}
\label{basic_info}
Naimplementovaná aplikácia sa pripája k požadovanému zdroju, stiahne ATOM alebo RSS 2.0 dáta, spracuje ich a vypíše užívateľovi na štandardný výstup. Pri HTTPS komunikácii, aplikácia podporuje SSL/TLS pre overenie certifikátu zdroju.

Požadovaný zdroj môže byť vložený pomocou jednej URL alebo pomocou súboru s viacerými URL pri spustení aplikácie.
Vždy sa vypíše názov zdroju a názvy jednotlivých príspevkov.
Užívateľ si môže nechať vypísať dátum publikovania, URI na konkrétny príspevok a autora príspevku.

\subsection*{Rozšírenie}
\begin{itemize}
  \item{Rôzne návratové hodnoty pri chybách, definovaných v {\tt../src/error.h}.}
  \item{Lokalizácia správ aplikácie do čestiny a angličtiny. Prepínač je v {\tt../src/error.h}.}
\end{itemize}

\subsection*{Obmedzenia}
\begin{itemize}
  \item{Kompilácia zdrojového kódu sa musí spraviť pomocou GNU Make.}
  \item{Aplikácia je vyvinutá pre UNIX OS.}
\end{itemize}

\subsection*{Kompilácia}
Predstavované riešenie obsahuje aj základné testy vstupných parametrov. Kompilácia testov nie je súčasť predvoleného kompilovania zdrojových súborov. Je ich možné pridať pridaním parametru {\tt test} k GNU make príkazu. Taktiež je možné testy skompilovať spolu s ostatnými zdrojovými súbormi aplikácie pomocou parametru {\tt all}. Príklady použitia sú v kapitole \ref{examples}.
Po kompilácii sa vytvoria spustiteľné súbory {\tt feedreader} a {\tt test} v priečinku s Makefile. Záleží či kompiluje len aplikácia, alebo len testy, alebo všetko.

\subsection*{Testy}
Implementácia obsahuje aj unit testy, ktoré testujú základné spracovanie vstupných parametrov.

\chapter{Návrh aplikácie}
\subsection*{Stromová štruktúra}
\dirtree{%
.0 root.
.1 bin.
.1 src.
.2 atom.c.
.2 atom.h.
.2 \dots.
.1 tests.
.2 examples.
.3 FIT-cert.
.3 \dots.
.2 \dots.
.1 Makefile.
.1 manual.pdf.
.1 README.md.
}
\hfill \break
\noindent Význam priečinkov aplikácie:
\begin{itemize}
  \item{priečinok {\tt bin} bude po kompilácii aplikácie obsahovať objektové súbory zdrojových súborov}
  \item{priečinok {\tt src} obsahuje zdrojové súbory implementácie aplikácie}
  \item{priečinok {\tt tests} obsahuje zdrojové súbory testov aplikácie a vzorové príklady}
\end{itemize}

\subsection*{Lokalizácia}
Nad rámec zadania bola naimplementovaná dvojjazyčná mutácia aplikácie. Lokalizácia poskytuje českú a anglickú jazykovú mutáciu.

Predvoleným jazykom je čestina a zmena je možná prenastavením hodnoty \textbf{LANG} v {\tt ../src/error.h} na 0 (nula). Toto nastavenie ovplyvňuje preklad častí výstupu pre užívateľa, ako napr. {\it Autor:} na {\it Author:} pri výpise autora článku. Ako ďalší vplyv tohto nastavenia je preklad všetkých chybových hláškok, viac v kapitole \ref{err_label}.

\chapter{Implementácia}
Naimplementovaná čítačka noviniek je napísaná v programovacom jazyku C. Pri vývoji aplikácie boli použité štandardné knižnice jazyka C ({\it stdio, stdlib, string, getopt, time\dots}), sieťové knižnice({\tt sys/*, netdb}), knižnica openssl pre TLS/SSL a knižnica libxml pre spracovanie XML odpovede.
\subsection*{Vstupné parametre}
Spracovanie užívateľského vstupu má na starosti {\tt ../src/parameters.c}. Kontroluje správnosť kombinácii vstupných parametrov a ich požadovaného správania, podľa zadania projektu.
\subsubsection{Možnosti}
\begin{itemize}
  \item{URL so schématom http alebo https}
  \item{-f {\it cesta-k-suboru-s-urls}}
  \item{-c {\it cesta-k-suboru-pertifikatu}}
  \item{-C {\it cesta-k-priečinku-pertifikatov}}
  \item{prepínače zobrazenia doplňujúcich dát článku: -a, -u, -T}
\end{itemize}
\subsubsection{Kombinácie}
Aplikácia \textbf{nepodporuje} kombinácie:
\begin{itemize}
  \item{URL a -f {\it cesta-k-suboru-s-urls}}
  \item{-c {\it cesta-k-suboru-pertifikatu} a -C {\it cesta-k-priečinku-pertifikatov}}
  \item{mnohonásobné použitie prepínačov alebo URL}
\end{itemize}

Validné príklady použitia sú k nahliadnutiu v kapitole \ref{examples}.

\subsection*{Spracovanie URL}
URL môže byť užívateľom vložená do aplikácie dvomi spôsobmi:
\begin{itemize}
  \item{pri spustení cez terminál}
  \item{vložený do súboru, ktorý je predaný aplikácii s parametrom {\tt -f}}
\end{itemize}

O spracovanie URL z príkazového riadku a súboru sa starajú funkcie súboru {\tt ../src/host.c}, konkrétne funkcie {\it parse\_url()} a {\it read\_urls()}. 

Spracovanie URL prebieha pomocou regulárneho výrazu\cite{regexMan}, ktorý bol navrhnutý tak, aby pri použití na URL vyhľadal a jednotlivé časti (skupiny) URL. A to:
\begin{itemize}
  \item{http:// alebo https://}
  \item{www.}
  \item{{\it domenovy-nazov-serveru.domenu-najvyssej-urovne}}
  \item{{\it :port}}
  \item{{\it cesta-k-suboru}}
\end{itemize}

\noindent Súbor pre parameter {\tt -f} môže obsahovať jednu URL na každom riadku. Ak má na začiatku znak '\#', tento riadok je braný ako komentár. Každý nekomentový riadok je spracovaný funkciou {\it parse\_url()}. Pri chybnom formáte URL, je na štandardný chybový výstup vypísaná chyba a aplikácia pokračuje ďalej vo vykonávaní implementácie.

\subsection*{TCP komunikácia}
Implementácia TCP časti sa nachádza v {\tt ../src/tcp\_communication.c}, ktorého podstata vychádza z prednášky a jej príkladov predmetu ISA, konkrétne prednášku docenta Matoušku, Pokrocilé programování sítí TCP/IP\cite{Matousek}. 

\subsubsection{HTTP a HTTPS}
HTTP a HTTPS komunikácia sú implementované v zdrojový súboroch 
\\{\tt ../src/http\_communication.c} a {\tt ../src/https\_communication.c}. Funkcie
\\({\tt send\_http\_request()} a {\tt send\_https\_request()}) oboch týchto komunikácií sú volané, po prevedí predošlých úkonov, napr. naviazenie TLS spojenia pri HTTPS. Zasielajú HTTP 1.0 dotaz na cieľovú destináciu. Inšpiráciou pre HTTP dotaz, po menšej úprave, bola časť kódu z referečnej knihy predmetu ISA\cite{Winkle}.
\subsubsection{Prijímanie a uloženie dát}
Prijímanie a uloženie je taktiež naimplementované v súboroch
\\{\tt ../src/http\_communication.c} a {\tt ../src/https\_communication.c}. Výsledným produktom, pri úspešnej odpovedi cieľového serveru a prijatí všetkých dát, je dynamicky alokovaný reťazec, ktorý sa ďalej spracováva.

Ako úspešnú odpoveď, implementácia považuje odpoveď so statusom {\tt 200}. Všetky ostatné\cite{ibmHttp} sú brané ako neúspešná odpoveď a spracovanie dát sa ukončuje. Užívateľ si môže v chybovej hláške nájsť.

Inšpiráciou pre prijímanie odpovedí z sielového serveru boli už spomínaná prednáška docenta Matoušku\cite{Matousek} a refrečná kniha predmetu ISA\cite{Winkle}.

Ukážky kódu z refrečnej knihy predmetu ISA\cite{Winkle}, po úpravách a rozšíreniach, boli taktiež použité, ako inšpirácia pre pre TLS komunikáciu a prácu s certifikátmi vo funkciách súboru {\tt ../src/https\_communication.c}.

\subsection*{Spracovanie XML}
Hlavné spracovanie XML odpovede sa deje v {\tt ../src/feed.c}, kde sa reťazec konvertuje do XML objektu pomocou funkcie knižnice {\it libxml}\cite{libxml}. Po úspešnom prekonvertovaní a nájdení koreňa XML, nasleduje spracovanie podľa typu štruktúry XML. Pri spracovaní sa vypíše názov zdroja a jednotlivé názvy článkov. 

Ak užívateľ zadal príslušné paramatri a odpoveď obsahuje také údaje, aplikácia vypíše informácie o autorovi, dátume vydania a URI článku. 

V prípade, že boli použité dodatočné parametre, články sú oddelené jedným prázdnym riadkom. Informácie o článku sú v poradí:
\begin{enumerate}
  \item{názov článku}
  \item{meno autora článku}
  \item{URI článku}
  \item{dátum publikovania článku}
\end{enumerate}

Ak nejaká dodatočná informácia chýba, alebo nebola vybraná, preskočí sa a vypíše sa ďalšia v poradí.

\subsubsection{Atom a RSS}
Spracovanie je implementované v {\tt ../src/atom.c} a v {\tt ../src/rss.c}.

\subsection*{Oštrenie chýb}
\label{err_label}
Každá chybová hláška má svoju českú a anglickú verziu. V zdrojových súboroch je volaná funkcia {\tt error\_msg()} s anglickou hláškou. 

Pred výpisom sa prejde matica dvojic. Jedna dvojica zodpovedá jednej z chybových možností. A teda pri prechode maticou sa nájde riadok vyskytnutej chyby a prepínačom {\tt LANG} sa určí stĺpec. Takto sa vyberá chybová hláška k výpisu pre užívateľa.

V matici dvojic sa nenachádzajú dva prípady chýb. Prvá je vypršanie času pri prijímaní dát v {\tt ../src/http\_communication.c} a 
\\{\tt ../src/https\_communication.c}. Tieto chyby si generujú správu samé a nastavujú globálne premenné návratovej hodnoty.

Druhou je zlá odpoveď na http dotaz rovnako v {\tt ../src/http\_communication.c} a 
\\{\tt ../src/https\_communication.c}. Na rozdiel od predošlej varianty, sa volá funkcia
\\{\tt error\_msg()}, ktorá iným spôsobom vyhodnocuje návratovú hodnotu aplikácie.

Ku matici dvojíc existuje matica návratových hodnôt, ktorá má na rovnakých pozíciach hodnotu, ktorá zodpovedá dvojici. Preto keď sa nájde riadok dvojice, ktorej chyba sa prejavila, tak vie aplikácia nájsť správnu návratovú hodnotu. 

Úspešný chod aplikácie vráti hodnotu 0. Pri chybách, aplikácia vráti príslušnú chybovú hodnotu, definovanú v {\tt ../src/error.h}.

Výpis návratových hodnôt:
\begin{itemize}
  \item{0 - úspech}
  \item{1 - všeobecná chyba}
  \item{2 - chyba pri kompilácii regulárného výrazu}
  \item{10 - žiadne vstupné parametre}
  \item{11 - neznámy parameter}
  \item{12 - niekoľkonásobné použitie rovnakého parametru}
  \item{13 - nepovolená kombinácia parametrov}
  \item{14 - viaceré zdroje cieľových destinácii}
  \item{15 - zlý formát URL}
  \item{16 - chýbajúca cesta k súboru na cieľovej destinácii}
  \item{17 - nezadaný zdroj cieľovej destinácie}
  \item{18 - chýbajúci/neznámy súbor s certifikátom}
  \item{19 - chýbajúci/neznámy priečinok s certifikátmi}
  \item{20 - nepodarilo sa preložiť URL na IP adresu}
  \item{21 - neplatná cieľová destinácia}
  \item{22 - vypršal čas na odpoveď}
  \item{23 - vytváranie schránky zlyhalo}
  \item{24 - nepodarilo sa naviazať spojenie s cieľovou destináciou}
  \item{30 - chyba pri čítaní súboru}
  \item{31 - súbor je prázdny alebo neobsahuje valídne URL}
  \item{40 - neúspešná odpoveď od cieľového serveru, http status > 200}
  \item{50 - chyba pri vytváraní CTX kontextu}
  \item{51 - chyba pri vytváraní SSL objketu}
  \item{52 - chyba pri naväzovaní SSL komunikácie}
  \item{53 - chyba pri identifikovaní serveru}
  \item{54 - nepodarilo sa získať certifikát cieľovej destinácie}
  \item{60 - XML odpoveď je prázdna}
  \item{61 - spracovanie XML stromu zlyhalo}
  \item{62 - chyba pri konvertovaní XML na XML objekt}
  \item{70 - chyba pri načítaní súboru certifikátu}
  \item{71 - chyba pri načítaní priečinku certifikátov}
  \item{72 - chyba pri nastavovaní predvoleného priečinku certifikátov}
  \item{73 - overenie certifikátu zlyhalo}
\end{itemize}

\chapter{Príklady použitia}
\label{examples}
\subsection*{Kompilácia}
Pri predvolenom GNU make.
\begin{itemize}
  \item{make -- kompilácia zdrojových súborov}
  \item{make test -- kompilácia zdrojových súborov testov}
  \item{make all -- kombilácia všetkých zdrojových súborov projektu}
\end{itemize}
\subsection*{Spustenie aplikácie}
\begin{itemize}
  \item{{\tt ./feedreader https://www.fit.vut.cz/fit/news-rss/}}
  \item{{\tt ./feedreader -T https://www.fit.vut.cz/fit/news-rss/ -a}}
  \item{{\tt ./feedreader https://www.fit.vut.cz/fit/news-rss/ -T -a -u}}
  \item{{\tt ./feedreader -f test/examples/hosts.txt -T -a -u}}
  \item{{\tt ./feedreader -u -c test/examples/FIT-cert https://www.fit.vut.cz/fit/news-rss/}}
  \item{{\tt ./feedreader -u -c test/examples/FIT-cert -f test/examples/hosts.txt}}
\end{itemize}
%===============================================================================

  
  % Kompilace po částech (viz výše, nutno odkomentovat)
  % Compilation piecewise (see above, it is necessary to uncomment it)
  %\subfile{projekt-01-uvod-introduction}
  % ...
  %\subfile{chapters/projekt-05-conclusion}


  % Pouzita literatura / Bibliography
  % ----------------------------------------------
\ifslovak
  \makeatletter
  \def\@openbib@code{\addcontentsline{toc}{chapter}{Literatúra}}
  \makeatother
  \bibliographystyle{bib-styles/Pysny/skplain}
\else
  \ifczech
    \makeatletter
    \def\@openbib@code{\addcontentsline{toc}{chapter}{Literatura}}
    \makeatother
    \bibliographystyle{bib-styles/Pysny/czplain}
  \else 
    \makeatletter
    \def\@openbib@code{\addcontentsline{toc}{chapter}{Bibliography}}
    \makeatother
    \bibliographystyle{bib-styles/Pysny/enplain}
  %  \bibliographystyle{alpha}
  \fi
\fi
  \begin{flushleft}
  \bibliography{projekt-20-literatura-bibliography}
  \end{flushleft}

  % vynechani stranky v oboustrannem rezimu
  % Skip the page in the two-sided mode
  \iftwoside
    \cleardoublepage
  \fi


  
\end{document}
